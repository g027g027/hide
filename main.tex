\documentclass{article}
\usepackage{graphicx} % Required for inserting images
\usepackage[swedish]{babel}
\usepackage{amsmath}
\usepackage{amssymb}
\usepackage{appendix}
\usepackage[margin=2.5cm]{geometry}
\usepackage{subcaption}

\title{Undersökning av egenskaper hos $G(n, p$)-slumpgrafer}

\author{Suheyla Bamukhtar, Gustav Jörgensen, Leo Lennartsson}
\date{datum September 2025}

\begin{document}

\maketitle


\begin{abstract}
Denna rapport undersöker egenskaper, såsom tröskelvärden och fördelningar, hos slumpgrafer enligt $G(n, p$)-modellen. Detta genomfördes med teoretisk analys och även simuleringar i Python där syftet var att besvara 5 frågor: uppkomsten av jättekomponenter, storleken på $p$ för att sannolikheten att slumpgrafen alltid ska vara sammanhängande då $n\rightarrow\infty$, fördelningen av trianglar respektive isolerade noder då $p=1/n$ och den största klickstorleken då $p=1/2$. Resultatet på fråga 1 tyder på bla bla bla. För fråga 2 blev storleken för $p_n = \frac{\log n}{n}$. För fråga 3.... För fråga 4 blev antalet isolerade noder poisson-fördelat med parameter $\lambda=n/e$. För fråga 5 blev resultatet att den största klicktstorleken i  $G(n, 1/2$) växer $2 \log_2(n)$. 
\end{abstract}

\newpage
\section{Inledning}

En graf $G=(V,E)$, där V är en mängd noder och E en mängd kanter mellan noderna. Slumpgrafen $G_n$ enligt $G(n, p)$-modellen, där varje möjlig kant mellan noder uppstår med sannolikhet $p$, fungerar som ett verktyg för att analysera egenskaper hos sådana grafer. Ett centralt intresse inom grafteori är att förstå vid vilka värden för $p$ som vissa egenskaper uppstår. 

Forskning om slumpgrafer har flera praktiska tillämpningar, bl. a. att modellera nätverk som internets webbgraf, människors sociala nätverk och neurala nätverk \cite{Tesliuc2017}.

Vi utgår från teoretiska resultat inom sannolikhetsteori, samt genomför egna simuleringar i Python för att illustrera resultaten. Projektets syfte är att både hitta och visualisera tröskelvärden, sannolikhetsfördelningar och beteenden hos slumpgrafer genom att försöka besvara följande frågor:
\begin{enumerate}
    \item Hur stort behöver $p = p_n$ vara som funktion av $n$ för att sannolikheten att den största sammanhängande komponenten innehåller minst $0.25n$ noder går mot $1$ då $n \to \infty$?
    \item Vilken storlek på $p=p_n$ krävs för att sannolikheten att slumpgrafen $G_n$ är sammanhängande går mot $1$ då $n \to \infty$?
    \item Om $p = 1/n$, vad är den asymptotiska fördelningen för antalet trianglar? (En triangel är en trippel noder $u, v, w$ sådan att alla tre kanterna $uv, uw$ och $vw$ finns med i $G_n$.)
    \item Vad är den asymptotiska fördelningen för antalet isolerade, alltså noder utan koppling till någon av grafens andra noder, då $p = 1/n$?
    \item Hur stor blir den största klicken, d.v.s den mängd där varje par noder har en kant mellan sig, då $p = 1/2$?
\end{enumerate}


\section{Metod}

För att besvara frågorna har vi använt oss av både teoretisk analys och simuleringar i Python.

\subsection{Fråga 1}
Vi ville undersöka vid vilka värden på $p=\tfrac{c}{n}$ som den största komponenten i $G(n,p)$ innehåller minst $0.25n$ noder \cite{jlr1999}.
Som bakgrund kan nämnas att det finns ett välkänt fasövergångsbeteende i slumpgrafer: när $c \leq 1$ består grafen mest av små komponenter, men när $c > 1$ uppträder en jättekomponent som omfattar en positiv andel av noderna.

För att generera slumpgraferna använde vi Batagelj--Brandes-algoritmen,
som gör det möjligt att skapa $G(n,p)$ i tid $O(n+m)$, där $m$ är antalet
kanter \cite{BatageljBrandes2005}. För varje graf beräknade vi den största komponentens storlek med BFS, och upprepade experimenten för flera värden på $c$ kring den kritiska punkten.

\subsection{Fråga 2}
För fråga 2 använder vi asymptotiska resultat från grafteori som säger att tröskelvärdet för sammanhängande graf i $G(n, p)$ är $p_n = \frac{\log n}{n}$ \cite{Tesliuc2017}. Vi simulerar slumpgrafer för $n=10,100,1000,2000$ och motsvarande $p=\frac{\log n}{n}$ för att undersöka när grafen blir sammanhängande.

\subsection{Fråga 3}
Vi undersökte hur antalet trianglar i $G(n,p)$ fördelar sig när $p=\tfrac{1}{n}$. 
Teorin säger att i denna skala blir trianglar sällsynta och nästan oberoende, vilket leder till att antalet trianglar $T$ konvergerar mot en Poisson-fördelning med parameter $\lambda=\tfrac{1}{6}$.

För att verifiera detta genomförde vi simuleringar i Python. Vi genererade slumpgrafer med $n=6000$ och $p=1/n$ med hjälp av Batagelj--Brandes-algoritmen. För varje graf räknade vi antalet trianglar med en effektiv metod baserad på grad-orientering. Vi upprepade försöken många gånger för att få en tillförlitlig fördelning.


\subsection{Fråga 4}
Vi undersökte hur antalet isolerade noder i $G(n,p)$ fördelas när $p=\tfrac{1}{n}$. Genom beräkningar fick vi att det teoretiska väntevärdet för antalet isolerade noder är $n/e$ då $n \to \infty$ och förväntas konvergera mot en Poisson-fördelning \cite{jlr1999}. För att verifiera detta genomfördes simuleringar i Python.

Med hjälp av en etablerad funktion (från Python-biblioteket NetworkX) som genererar slumpgrafer enligt $G(n, p)$-modellen, genererades $G(n,1/n)$ där n=1000 och 500 simuleringar genomfördes. Antalet isolerade noder i varje graf sparades och plottades i ett histogram. 

\subsection{Fråga 5}
För fråga 5 har vi beräknat storleken på den största klicken vid fast sannolikhet $p = 1/2$, där det är känt att den största klicken i $G(n, 1/2)$ är ungefär $2 \log_2(n)$ \cite{bollobas2001}. Vi har verifierat detta genom simuleringar där vi genererade $G(n, 1/2)$-grafer med varierande n (många simuleringar av samma n genomfördes), hittade de maximala klickar i varje graf och slutligen gjordes en regressionslinje där det genomsnittliga största klickstorleken för olika n plottades.


\section{Resultat}

\subsection{Fråga 1}
Simuleringarna visar tydligt fasövergången: för $c$ strax över 1 börjar den största komponenten snabbt växa. 
Vid värden kring $c \approx 1.15$ observerade vi att den största komponenten i genomsnitt omfattar ungefär $25\%$ av alla noder. 
För större $c$ fortsätter andelen att växa. Resultaten från simuleringarna överensstämmer väl med de teoretiska förväntningarna.

\begin{center}
\includegraphics[width=0.75\linewidth]{output.png}
\end{center}

\subsection{Fråga 2}
I fråga 2 simulerade vi slumpgrafer enligt $G(n,p)$-modellen för olika värden på 
$n$, där $p$ sattes till det teoretiska tröskelvärdet 
$p_n=\frac{\log{n}}{n}$. För varje värde på $n$ genererades ett antal grafer, och vi mätte andelen som var sammanhängande.

I Figur~\ref{sammanhängandegrafer} ser vi att sannolikheten går mot 1 då $n \rightarrow \infty$. Redan vid $n=100$ kan vi observera den skarpa övergången där sannolikheten går mot 1 för detta p. Vi ser att tröskelvärdet närmar sig 0 då $n$ växer ($\frac{\log n}{n}\rightarrow0, n\rightarrow\infty$) alltså går sannolikheten mot 1 då $n\rightarrow\infty$.

\begin{figure}
    
    \begin{subfigure}{0.475\linewidth}
    \includegraphics[width=\linewidth]{sammanhängandegraf10.png}
    \caption{$n=10$}
    \end{subfigure}
\hfill
    \begin{subfigure}{0.475\linewidth}
    \includegraphics[width=\linewidth]{sammanhängandegraf100.png}
    \caption{$n=100$}
    \end{subfigure}
    
\medskip

    \begin{subfigure}{0.475\linewidth}
    \includegraphics[width=\linewidth]{sammanhängandegraf1000.png}
    \caption{$n=1000$}
    \end{subfigure}
\hfill
    \begin{subfigure}{0.475\linewidth}
    \includegraphics[width=\linewidth]{sammanhängandegraf2000.png}
    \caption{$n=2000$}
    \end{subfigure}

    \caption{Sannolikheten att $G(n,p)$ är sammanhängande för olika $n$.}
    \label{sammanhängandegrafer}
\end{figure}

\subsection{Fråga 3}
Resultaten från simuleringarna visar att antalet trianglar oftast är 0 eller 1, och mer sällan 2 eller fler. Histogrammet från våra försök stämmer väl överens med sannolikhetsmassfunktionen för en Poissonfördelning med $\lambda=\tfrac{1}{6}$.
\begin{figure}
    \centering
    \includegraphics[width=0.75\linewidth]{output3.png}
    \caption{}
    \label{}
\end{figure}
Det empiriska medelvärdet och variansen för antalet trianglar låg mycket nära $\tfrac{1}{6}$, vilket stärker slutsatsen att fördelningen är Poissonliknande.

\subsection{Fråga 4} 
\begin{figure}[h!]
    \centering
    \includegraphics[width=0.5\linewidth]{F4.png}
    \caption{Fördelningen av antalet isolerade noder i $G(n, 1/n)$ då n=1000.}
    \label{figf4}
\end{figure}
Histogrammet i figur~\ref{figf4} visar den empiriska fördelningen av G(n, 1/n) för n=1000 över 500 simuleringar. Fördelningen centreras kring ett specifikt värde, ungefär 370, och är någorlunda klockformad samt asymmetrisk. 


\subsection{Fråga 5}
\begin{figure}[h!]
    \centering
    \includegraphics[width=0.5\linewidth]{image.png}
    \caption{Tillväxten av den största klickstorleken i $G(n, 1/2)$.}
    \label{figf5}
\end{figure}
Figur~\ref{figf5} visar hur de empiriska punkterna (genomsnittlig klickstorlek) följer en logaritmisk trendlinje där lutningen är $\approx 1.70$, vilket till viss del stödjer att den största klickstorleken i $G(n, 1/2)$ växer $2 \log_2(n)$.

\section{Diskussion}

\subsection{Fråga 1}

Våra simuleringar bekräftar den teoretiska bilden av slumpgrafernas fasövergång: 
för små $c$ saknas en jättekomponent, men när $c$ ökar förbi ett tröskelvärde uppträder en komponent som växer linjärt med $n$. 
Vi kan konstatera att runt $c \approx 1.15$ blir denna komponent ungefär en fjärdedel av hela grafen. 
Små avvikelser mellan simulering och teori kan förklaras av ändliga storlekseffekter, men trenden är tydlig även för måttligt stora $n$.

\subsection{Fråga 2}

Resultaten för fråga 2 stämmer väl överens med tidigare teoretiska förutsägelser om slumpgrafer. Det observerade tröskelvärdet vid $p_n=\frac{\log{n}}{n}$
visar den tydliga övergången från icke-sammanhängande till sammanhängande grafer. Denna övergång går mot noll då $n\rightarrow\infty$ och sannolikheten att slumpgrafen är sammanhängande går mot 1.

\subsection{Fråga 3}
Simuleringarna bekräftar den teoretiska förutsägelsen att antalet trianglar i $G(n,1/n)$ konvergerar mot en Poissonfördelning med parameter $\lambda=\tfrac{1}{6}$. Detta illustrerar ett generellt fenomen i slumpgrafer: när sannolikheten väljs i en kritisk skala (här $1/n$) blir små strukturer som trianglar sällsynta men följer en enkel slumpmodell.

Små avvikelser i frekvensen av 2 eller fler trianglar kan tillskrivas ändliga storlekseffekter och slumpvariation i simuleringarna. För större $n$ skulle dessa skillnader minska ytterligare. Våra resultat visar att redan vid $n=6000$ syns Poissonbeteendet tydligt.

\subsection{Fråga 4}
Det teoretiska väntevärdet för antalet isolerade noder i en $G(1000,1/1000)$, $1000/e \approx 367.88$, överensstämmer väl med det mest frekventa antalet isolerade noder i histogrammet. Resultatet stödjer den teoretiska förutsägelsen att den  asymptotiska fördelningen för antalet isolerade noder är en Poisson-fördelning med parameter $\lambda=n/e$.

Tillförlitligheten av resultat för vad den asymptotiska fördelningen för antalet isolerade noder blir, hade kunnat öka genom att testa simuleringar för ökande n, t.ex n=500, n=1000 och n=2000. 

\subsection{Fråga 5}
Simuleringarna för att bestämma den största klicken utnyttjade NetworkX-funktionen find\_cliques som var effektiv men vi stötte ändå på problem eftersom att hitta maximal klickstorlek är ett NP-svårt problem. Ursprungsidén var n=1000 men detta var ej hanterbart för datorn; genom att helt enkelt testa fick vi fram att 200 var det största, hanterbara värdet för n. 

Mer tillförlitliga resultat skulle kunna erhållas om man testar för större n-värden och förstås flera simuleringar.  

\bibliographystyle{plain} 
\bibliography{referenser}

\newpage
\appendix
\section{Appendix}

Fråga 1 \& 3
\begin{verbatim}
# Löser uppgift 1 och 3 med Python.
# - Uppgift 1: när blir största komponenten ≥ 0.25 n (p_n = c/n). Jämför simulering med teori.
# - Uppgift 3: antal trianglar när p = 1/n. Jämför mot Poisson(1/6).

import math
import numpy as np
import pandas as pd
from collections import deque
import matplotlib.pyplot as plt
from pathlib import Path
from typing import List, Tuple, Set

# --------------------------
# Hjälpfunktioner
# --------------------------
def fast_gnp_undirected(n: int, p: float, rng: np.random.Generator) -> List[Tuple[int, int]]:
    """
    Skapar kanter i G(n,p) med Batagelj–Brandes (förväntad O(m)).
    Returnerar lista av (u,v) med u < v.
    """
    if p <= 0.0 or n <= 1:
        return []
    if p >= 1.0:
        return [(u, v) for u in range(n) for v in range(u + 1, n)]

    lp = math.log(1.0 - p)
    v = 1
    w = 0
    edges = []
    # Hoppa över icke-kanter geometriskt och lägg till nästa kant
    while v < n:
        r = rng.random()
        skip = 0 if r == 0.0 else int(math.floor(math.log(1.0 - r) / lp))
        w = w + skip + 1
        while w >= v and v < n:
            w = w - v
            v = v + 1
        if v < n:
            edges.append((w, v))
    return edges

def edges_to_adj(n: int, edges: List[Tuple[int, int]]) -> List[Set[int]]:
    """Bygger grannskapslista från kantlista."""
    adj = [set() for _ in range(n)]
    for u, v in edges:
        adj[u].add(v)
        adj[v].add(u)
    return adj

def largest_component_size(adj: List[Set[int]]) -> int:
    """Storleken på största komponenten (BFS)."""
    n = len(adj)
    seen = [False] * n
    best = 0
    for s in range(n):
        if not seen[s]:
            q = deque([s])
            seen[s] = True
            size = 0
            while q:
                u = q.popleft()
                size += 1
                for v in adj[u]:
                    if not seen[v]:
                        seen[v] = True
                        q.append(v)
            if size > best:
                best = size
    return best

def fixed_point_beta(c: float, tol: float = 1e-12, max_iter: int = 10_000) -> float:
    """
    Löser β = 1 - exp(-cβ) (storleken på jättekomponenten som andel).
    Ger 0 för c ≤ 1, annars lösningen i (0,1).
    """
    if c <= 1.0:
        return 0.0
    b = 0.5
    for _ in range(max_iter):
        nb = 1.0 - math.exp(-c * b)
        if abs(nb - b) < tol:
            return nb
        b = nb
    return b

# --------- Snabb triangelräkning (grad-orientering) ----------
def count_triangles_degree_oriented(adj: List[Set[int]]) -> int:
    """
    Räknar trianglar exakt:
    - Orientera kant u->v om deg(u)<deg(v) eller (deg lika och u<v).
    - T = sum_u sum_{v i F(u)} |F(u) ∩ F(v)|.
    """
    n = len(adj)
    deg = [len(adj[u]) for u in range(n)]
    F = [set() for _ in range(n)]
    for u in range(n):
        du = deg[u]
        for v in adj[u]:
            dv = deg[v]
            if (du < dv) or (du == dv and u < v):
                F[u].add(v)
    tri = 0
    for u in range(n):
        Fu = F[u]
        if not Fu:
            continue
        for v in Fu:
            tri += len(Fu & F[v])
    return tri

# --------------------------
# Uppgift 1: största komponenten ≥ 0.25 n
# --------------------------
rng = np.random.default_rng(12345)

c_star = -math.log(0.75) / 0.25           # tröskel där β = 0.25
c_grid = np.linspace(0.8, 1.4, 121)       # för teorikurva
beta_theory = np.array([fixed_point_beta(float(c)) for c in c_grid])

# Simuleringsparametrar (justera vid behov)
n_p1 = 30000
c_values = [1.05, 1.10, c_star, 1.16, 1.20, 1.30]
replicates = 10

sim_rows = []
for c in c_values:
    p = c / n_p1
    for r in range(replicates):
        edges = fast_gnp_undirected(n_p1, p, rng)
        adj = edges_to_adj(n_p1, edges)
        L = largest_component_size(adj)
        sim_rows.append({
            "c": c,
            "p": p,
            "replicate": r + 1,
            "largest_component_frac": L / n_p1
        })

p1_df = pd.DataFrame(sim_rows)
# Stabil kolumnsättning: platt DataFrame
p1_summary = (
    p1_df
    .groupby("c", as_index=False)["largest_component_frac"]
    .agg(mean="mean", std="std", min="min", max="max")
)
# Lägg till teori
p1_summary["beta_theory"] = p1_summary["c"].apply(lambda c: fixed_point_beta(float(c)))

# Figur: teori vs simulering
plt.figure(figsize=(7, 4.5))
plt.plot(c_grid, beta_theory, label="Teoretisk β(c)")
plt.scatter(p1_summary["c"], p1_summary["mean"], label="Simulerad andel (medel)")
plt.axhline(0.25, linestyle="--", label="Mål: 0.25")
plt.axvline(c_star, linestyle="--", label=f"c* ≈ {c_star:.6f}")
plt.xlabel("c i p_n = c / n")
plt.ylabel("Andel i största komponenten")
plt.title("Uppgift 1: jättekomponentens storlek")
plt.legend()
fig1_path = Path("uppgift1_giant_component.png")
plt.tight_layout()
plt.savefig(fig1_path)
plt.close()

# Spara data
p1_raw_path = Path("uppgift1_simulation_raw.csv")
p1_summary_path = Path("uppgift1_simulation_summary.csv")
p1_df.to_csv(p1_raw_path, index=False)
p1_summary.to_csv(p1_summary_path, index=False)

# --------------------------
# Uppgift 3: trianglar för p = 1/n
# --------------------------
rng = np.random.default_rng(54321)

n_p3 = 6000
p_p3 = 1.0 / n_p3
trials = 600   # öka för jämnare kurva

tri_counts = []
for t in range(trials):
    edges = fast_gnp_undirected(n_p3, p_p3, rng)
    adj = edges_to_adj(n_p3, edges)
    tri = count_triangles_degree_oriented(adj)
    tri_counts.append(tri)

p3_df = pd.DataFrame({"trial": np.arange(1, trials+1), "triangles": tri_counts})

# Empirisk fördelning
emp_counts = p3_df["triangles"].value_counts().sort_index()
emp_probs = (emp_counts / trials)

# Poisson(1/6) för jämförelse
lam = 1.0 / 6.0
k_max = max(5, int(emp_counts.index.max()))
k_plot = np.arange(0, k_max + 1)
pois_pmf = np.array([math.exp(-lam) * (lam**k) / math.factorial(k) for k in k_plot])

# Figur: empiri vs Poisson
plt.figure(figsize=(7, 4.5))
heights = emp_probs.reindex(k_plot, fill_value=0).to_numpy()
plt.bar(k_plot, heights, alpha=0.6, label="Empirisk sannolikhet")
plt.plot(k_plot, pois_pmf, marker="o", label="Poisson(1/6)")
plt.xlabel("Antal trianglar T")
plt.ylabel("Sannolikhet")
plt.title("Uppgift 3: trianglar vid p = 1/n")
plt.legend()
fig3_path = Path("uppgift3_triangles_distribution.png")
plt.tight_layout()
plt.savefig(fig3_path)
plt.close()

# Spara data
p3_raw_path = Path("uppgift3_triangle_counts.csv")
p3_df.to_csv(p3_raw_path, index=False)

# Utskrift av nyckelresultat
emp_mean = p3_df["triangles"].mean()
emp_var = p3_df["triangles"].var(ddof=0)

print("\nUppgift 1 – Nyckelresultat")
print(f"c* (β=0.25) = {c_star:.12f}")
print(p1_summary.to_string(index=False))

print("\nUppgift 3 – Nyckelresultat")
print(f"Antal försök: {trials}, n = {n_p3}, p = 1/n = {p_p3:.6g}")
print(f"Empiriskt medel(T): {emp_mean:.6f} (teori 1/6 ≈ {lam:.6f})")
print(f"Empirisk varians(T): {emp_var:.6f} (teori 1/6 ≈ {lam:.6f})")

print("\nFiler skapade:")
print(f"- Figur (Uppgift 1): {fig1_path}")
print(f"- Data (Uppgift 1, rå): {p1_raw_path}")
print(f"- Data (Uppgift 1, sammanfattning): {p1_summary_path}")
print(f"- Figur (Uppgift 3): {fig3_path}")
print(f"- Data (Uppgift 3, rå): {p3_raw_path}")

\end{verbatim}

Fråga 2
\begin{verbatim}
import numpy as np
import matplotlib.pyplot as plt
from collections import deque

# generera grannmatris för G(n, p)
def generate_random_graph(n, p):
    adj_matrix = np.random.rand(n, n) < p
    adj_matrix = np.triu(adj_matrix, 1)
    adj_matrix += adj_matrix.T  # symmetrisk matris
    return adj_matrix.astype(int)

# kolla sammanhängande
def is_connected(adj_matrix):
    n = adj_matrix.shape[0]
    visited = np.zeros(n, dtype=bool)
    queue = deque([0])
    visited[0] = True

    while queue:
        node = queue.popleft()
        neighbors = np.where(adj_matrix[node] == 1)[0]
        for neighbor in neighbors:
            if not visited[neighbor]:
                visited[neighbor] = True
                queue.append(neighbor)
    
    return visited.all()

# estimera sannolikheten för sammanhängande graf
def is_connected_prob(n, ps, trials=100):
    probs = []
    for p in ps:
        count = 0
        for _ in range(trials):
            adj_matrix = generate_random_graph(n, p)
            if is_connected(adj_matrix):
                count += 1
        probs.append(count / trials)
    return probs

# parametrar (n = 10,100,1000,2000)
n = 100
log_n_over_n = np.log(n) / n
ps = np.linspace(0.5 * log_n_over_n, 2 * log_n_over_n, 20)

# simulering
probs = is_connected_prob(n, ps)

# plot av resultat
plt.plot(ps, probs, marker='o')
plt.axvline(log_n_over_n, color='r', linestyle='--', label='log(n)/n')
plt.xlabel("p")
plt.ylabel("Sannolikhet för sammanhängande graf")
plt.title(f"Sannolikhet att G(n, p) är sammanhängande (n={n})")
plt.legend()
plt.grid(True)
plt.show()
\end{verbatim}


\textbf{Fråga 4}
\begin{verbatim}
import networkx as nx
import numpy as np
import matplotlib.pyplot as plt
import seaborn as sns

#fkn som genererar G(n, 1/n) och sparar antalet isolerade noder i varje simulering. 
def simulate_isolated_nodes(n, num_simulations=500):
    isolated_counts = []

    for _ in range(num_simulations):
        G = nx.erdos_renyi_graph(n, 1/n)
        isolated = sum(1 for node in G.nodes if G.degree[node] == 0)
        isolated_counts.append(isolated)

    return isolated_counts

# Plotta och rita histogrammet då n=1000
n = 1000  
isolated = simulate_isolated_nodes(n)

sns.histplot(isolated, kde=True)
plt.title(f"Fördelning av isolerade noder för n = {n}, p = 1/n")
plt.xlabel("Antal isolerade noder")
plt.ylabel("Frekvens")
plt.show()

\end{verbatim}

Fråga 5
\begin{verbatim}
import networkx as nx
import numpy as np
import matplotlib.pyplot as plt
import seaborn as sns

def simulate_largest_clique(n, num_simulations=200):
    from networkx.algorithms.clique import find_cliques
    max_clique_sizes = []

    for _ in range(num_simulations):
        G = nx.erdos_renyi_graph(n, 0.5)
        max_clique = max((len(c) for c in find_cliques(G)), default=0)
        max_clique_sizes.append(max_clique)

    return max_clique_sizes

# Kör simuleringar
n = 200  # n=1000, n=500 krascha/tog för lång tid
clique_sizes = simulate_largest_clique(n)

#försöka identifiera ett mönster i storleken av klick baserat på n.
def average_max_clique (n_values, num_simulations=200):
    averages = []

    for n in n_values:
        clique_sizes = simulate_largest_clique(n, num_simulations)
        avg = np.mean(clique_sizes)
        averages.append(avg)

    return averages

n_values = np.arange(20, 201, 10)  # n= 20, 30, ..., 200
avg_clique_sizes = average_max_clique (n_values)

# Logaritmisk regression: log2(n)
from scipy.stats import linregress

log_n = np.log2(n_values)
slope, intercept, r_value, _, _ = linregress(log_n, avg_clique_sizes)

# Plotta resultat
plt.plot(n_values, avg_clique_sizes, 'o', label="Empirisk")
plt.plot(n_values, slope * log_n + intercept, 'r--', label=f"Approx ≈ {slope:.2f} * log_2(n) + {intercept:.2f}")
plt.xlabel("antal noder")
plt.ylabel("Genomsnittlig största klickens storlek")
plt.title("Tillväxt av största klicken för olika n (p = 1/2)")
plt.legend()
plt.grid(True)
plt.show()
\end{verbatim}

\end{document}
